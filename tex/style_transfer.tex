\documentclass[journal, 12pt]{IEEEtran}
%
% If IEEEtran.cls has not been installed into the LaTeX system files,
% manually specify the path to it like:
% \documentclass[journal]{../sty/IEEEtran}

% *** CITATION PACKAGES ***
%
\usepackage{cite}

% *** GRAPHICS RELATED PACKAGES ***
%
\usepackage[pdftex]{graphicx}
\graphicspath{{resources/}}
\DeclareGraphicsExtensions{.pdf,.jpeg,.JPG,.png,.eps}

% *** ALIGNMENT PACKAGES ***
%
\usepackage{array}

\usepackage{mdwmath}
%\usepackage{mdwtab}

\usepackage{mathtools}


% *** FLOAT PACKAGES ***
%
\usepackage{fixltx2e}


% *** PDF, URL AND HYPERLINK PACKAGES ***
%
\usepackage{url}

%TODO: correct bad hyphenation here
%\hyphenation{op-tical net-works semi-conduc-tor}

\usepackage[numbers]{natbib}
\bibliographystyle{IEEEtranN}


%Support of umlaut ä, ö, ü
\usepackage[utf8]{inputenc}

%\usepackage{caption} 
%\captionsetup[table]{skip=10pt}

%-----------------------------------------------------
% Variablen definieren

\def \myTitle {Tensorflow Implementation of Deep Photo Style Transfer} %TODO
\def \myAuthor {Paul Sanzenbacher, Sebastian Penhou\"{e}t}
\def \myUniversity {University of Tübingen}
\def \myClass {Computer Graphics Practical Course}
\def \myDepartment {Department of Computer Science}
\def \myKeywords {style transfer, tensorflow} %TODO

%-----------------------------------------------------
%Eigene Einstellungen

\usepackage[nohyperlinks, nolist]{acronym}

% Notizen. Einsatz mit \todo{Notiz} oder \todo[inline]{Notiz}. 
\usepackage[obeyFinal,backgroundcolor=yellow,linecolor=black]{todonotes}
% Alle Notizen ausblenden mit der Option "final" in \documentclass[...] oder durch das auskommentieren folgender Zeile
% \usepackage[disable]{todonotes}

% Kommentarumgebung. Einsatz mit \comment{}. Alle Kommentare ausblenden mit dem Auskommentieren der folgenden und dem aktivieren der nächsten Zeile.
\newcommand{\comment}[1]{\par {\bfseries \color{blue} #1 \par}} %Kommentar anzeigen
% \newcommand{\comment}[1]{} %Kommentar ausblenden

\usepackage{xcolor}

\usepackage{hyperref}

\definecolor{colorOfLink}{named}{black}

\hypersetup{%
	pdfencoding=auto,
	pdftitle={\myTitle}, 
	pdfauthor={\myAuthor, \myUniversity}, 
	pdfsubject={\myClass, \myDepartment}, 
	pdfkeywords={\myKeywords},
	pdfcreator={pdflatex - TeXstudio}, 
	pdfpagemode=UseOutlines, 
	pdfdisplaydoctitle=true, 
	pdflang={English}, 
	colorlinks=true, 
	linkcolor=colorOfLink, 
	citecolor=colorOfLink, 
	filecolor=colorOfLink, 
	menucolor=colorOfLink, 
	urlcolor=colorOfLink, 
	linktocpage=true, 
	bookmarksnumbered=true
}

% Workaround um Fehler in Hyperref, muss hier stehen bleiben
\usepackage{bookmark} %nur ein latex-Durchlauf für die Aktualisierung von Verzeichnissen nötig

\newcommand*{\fullref}[1]{\hyperref[{#1}]{\autoref*{#1} \nameref*{#1}}}

\providecommand*{\lstlistingautorefname}{listing}

%\newcommand{\citepage}[2][0]{\cite[p.~#1]{#2}}

\usepackage{listings}
\lstset{%
	breaklines=true,
	basicstyle=\ttfamily
}

\lstMakeShortInline[columns=fixed]|

\usepackage{scrextend}

\usepackage{cleveref}

\usepackage{pgfplots}
\pgfplotsset{every axis plot/.append style={very thick}}
\pgfplotsset{
	cycle list={gray\\black\\lightgray\\},
}


\usepackage{setspace}
%\onehalfspacing

\usepackage{siunitx}

\usepackage{tabu}
\tabulinesep=1.2mm

\usepackage{tikz}
\usetikzlibrary{shapes,arrows}

\usepackage{caption}
\captionsetup{justification=centering}

%-----------------------------------------------------

\begin{document}
%
% paper title
% Titles are generally capitalized except for words such as a, an, and, as,
% at, but, by, for, in, nor, of, on, or, the, to and up, which are usually
% not capitalized unless they are the first or last word of the title.
% Linebreaks \\ can be used within to get better formatting as desired.
% Do not put math or special symbols in the title.
\title{\myTitle}
%
%
% author names and IEEE memberships
% note positions of commas and nonbreaking spaces ( ~ ) LaTeX will not break
% a structure at a ~ so this keeps an author's name from being broken across
% two lines.
% use \thanks{} to gain access to the first footnote area
% a separate \thanks must be used for each paragraph as LaTeX2e's \thanks
% was not built to handle multiple paragraphs
%

\author{\myAuthor \\ \IEEEmembership{\myUniversity}}

% note the % following the last \IEEEmembership and also \thanks - 
% these prevent an unwanted space from occurring between the last author name
% and the end of the author line. i.e., if you had this:
% 
% \author{....lastname \thanks{...} \thanks{...} }
%                     ^------------^------------^----Do not want these spaces!
%
% a space would be appended to the last name and could cause every name on that
% line to be shifted left slightly. This is one of those "LaTeX things". For
% instance, "\textbf{A} \textbf{B}" will typeset as "A B" not "AB". To get
% "AB" then you have to do: "\textbf{A}\textbf{B}"
% \thanks is no different in this regard, so shield the last } of each \thanks
% that ends a line with a % and do not let a space in before the next \thanks.
% Spaces after \IEEEmembership other than the last one are OK (and needed) as
% you are supposed to have spaces between the names. For what it is worth,
% this is a minor point as most people would not even notice if the said evil
% space somehow managed to creep in.



% The paper headers
\markboth{\myClass, \myDepartment}%
{\myAuthor: \myTitle}
% The only time the second header will appear is for the odd numbered pages
% after the title page when using the twoside option.


% use for special paper notices
%\IEEEspecialpapernotice{(Invited Paper)}

% make the title area
\maketitle

% As a general rule, do not put math, special symbols or citations
% in the abstract or keywords.
\begin{abstract}
	%\todo[inline]{Necessary?}
\end{abstract}

% Note that keywords are not normally used for peerreview papers.
\begin{IEEEkeywords}
	\myKeywords.
\end{IEEEkeywords}

\section{Introduction} \label{sec:intro}

\IEEEPARstart{T}{he} 



%--------------------------------------------------


\begin{acronym}
	\acro{ann}[ANN]{artificial neural network}
\end{acronym}

% if have a single appendix:
%\appendix[Proof of the Zonklar Equations]
% or
%\appendix  % for no appendix heading
% do not use \section anymore after \appendix, only \section*
% is possibly needed

% use appendices with more than one appendix
% then use \section to start each appendix
% you must declare a \section before using any
% \subsection or using \label (\appendices by itself
% starts a section numbered zero.)
%


%\appendices
%\section{Proof of the First Zonklar Equation}
%Appendix one text goes here.

% you can choose not to have a title for an appendix
% if you want by leaving the argument blank
%\section{}
%Appendix two text goes here.


% use section* for acknowledgment
%\section*{Acknowledgment}


%The authors would like to thank...


% Can use something like this to put references on a page
% by themselves when using endfloat and the captionsoff option.
%\ifCLASSOPTIONcaptionsoff
%  \newpage
%\fi



% trigger a \newpage just before the given reference
% number - used to balance the columns on the last page
% adjust value as needed - may need to be readjusted if
% the document is modified later
%\IEEEtriggeratref{8}
% The "triggered" command can be changed if desired:
%\IEEEtriggercmd{\enlargethispage{-5in}}

% references section

% argument is your BibTeX string definitions and bibliography database(s)
\bibliography{lit}
%

\end{document}